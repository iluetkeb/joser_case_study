%% joser_cs_template.tex
%% V1
%% 2014/02/24
%% by Ingo Lütkebohle
%%
%%*************************************************************************
%% Legal Notice:
%% This document is offered as-is without any warranty either expressed or
%% implied; without even the implied warranty of MERCHANTABILITY or
%% FITNESS FOR A PARTICULAR PURPOSE!
%% User assumes all risk.
%% In no event shall JOSER or any contributor to this code be liable for
%% any damages or losses, including, but not limited to, incidental,
%% consequential, or any other damages, resulting from the use or misuse
%% of any information contained here.
%%
%% This work is distributed under the LaTeX Project Public License (LPPL)
%% ( http://www.latex-project.org/ ) version 1.3, and may be freely used,
%% distributed and modified. A copy of the LPPL, version 1.3, is included
%% in the base LaTeX documentation of all distributions of LaTeX released
%% 2003/12/01 or later.
%% Retain all contribution notices and credits.
%% ** Modified files should be clearly indicated as such, including  **
%% ** renaming them and changing author support contact information. **
%%
%%*************************************************************************

\documentclass[10pt,journal,compsoc]{joser1}
\usepackage[utf8]{inputenc}

% *** GRAPHICS RELATED PACKAGES ***
%
\ifCLASSINFOpdf
   \usepackage[pdftex]{graphicx}
\else
   \usepackage[dvips]{graphicx}
\fi


%%%%%%%%%%%%%%%%%%%%%%%%%%%%%%%%%%%%%%%%%%%%%%%%%%%%%%%%%%%%%%%%%%%%%%%
%%%%%%%%%%%%%%%%%%%%%%% will be inserted by the editor %%%%%%%%%%%%%%%%
%%%%%%%%%%%%%%%%%%%%%%%%%%%%%%%%%%%%%%%%%%%%%%%%%%%%%%%%%%%%%%%%%%%%%%%
\journalnumber{1}                       %will be inserted by the editor
\journalvolume{1}                       %will be inserted by the editor
\journalmonth{February}                 %will be inserted by the editor
\journalyear{2014}                      %will be inserted by the editor
\articlefirstpage{123}                  %will be inserted by the editor
\articlelastpage{126}                   %will be inserted by the editor
\setcounter{page}{123}                  %will be inserted by the editor
%%%%%%%%%%%%%%%%%%%%%%%%%%%%%%%%%%%%%%%%%%%%%%%%%%%%%%%%%%%%%%%%%%%%%%%

\copyrightauthor{F. Author, S. Author, T. Author}
\headoddname{F. A. AUTHOR et al./ Preparation of Papers for {\sl Journal of Software Engineering for Robotics}}%

% correct bad hyphenation here
\hyphenation{op-tical net-works semi-conduc-tor}


\begin{document}
% paper title
\title{Preparation of Papers for the \emph{Case Study and Experience Reports} section of the \\\vskip 0.3\baselineskip Journal of Software Engineering for Robotics}

\author{
First-Aa AUTHOR$^{1,*}$
\qquad
Second-Bb AUTHOR$^{1,2}$
\qquad
Third AUTHOR$^2$

%%%%%%%%%%%%%%%%%%%%%%%%%%%%%%%%%%%%%%%%%%%%%%%%%%%%%%%%%%%%%%%%%%%%%%%
%%%%%%%%%%%%%%%%%%%%%%% will be inserted by the editor %%%%%%%%%%%%%%%%
%%%%%%%%%%%%%%%%%%%%%%%%%%%%%%%%%%%%%%%%%%%%%%%%%%%%%%%%%%%%%%%%%%%%%%%
\thanks{{\bf Regular paper} -- Manuscript received Month Day, Year;
revised Month Day, Year.}
%%%%%%%%%%%%%%%%%%%%%%%%%%%%%%%%%%%%%%%%%%%%%%%%%%%%%%%%%%%%%%%%%%%%%%%


\IEEEcompsocitemizethanks{\IEEEcompsocthanksitem This work was
supported by xxxxxxxx (No.xxxxxxxx) (sponsor and financial support
acknowledgment goes here).\protect\\

\IEEEcompsocthanksitem Authors retain copyright to their papers
and grant JOSER unlimited rights to publish the paper
electronically and in hard copy. Use
of the article is permitted as long as the author(s) and the journal are properly
acknowledged.}

} % end author

\address{
$^1$ Department of Computer Science, University of Bergamo,
Dalmine 24044, Italy\\
$^2$ Department of Computing and Electronic Systems, University of
Essex Colchester CO43SQ, UK }


% The paper headers
\markboth

\IEEEcompsoctitleabstractindextext{%
\begin{abstract}
An abstract should be a concise summary of the
significant items in the paper, including the results and
conclusions. It should be not more than about 500 words. Define all nonstandard symbols,
abbreviations and acronyms used in the abstract. Do not cite
references in the abstract.

A list of significant keywords (2 minimum, 5 recommended) should
be included in the first page of each submitted paper. Keywords
must be chosen from those predefined within the IEEE RAS subject
areas (\url{http://www.ieee-ras.org/uploads/tro/T-RO\_Keywords.pdf}) or
the IEEE CS subject area
(\url{http://www.computer.org/portal/pages/ieeecs/publications/author/keywords/ACMtaxonomy.html}).

\end{abstract}

\begin{IEEEkeywords}
Computer Society, JOSER, journal, \LaTeX, paper, template.
\end{IEEEkeywords}}


% make the title area
\maketitle


\section*{Preface for the template}
% The very first letter is a 2 line initial drop letter followed
% by the rest of the first word in caps.
\IEEEPARstart{T}{he} {\sl section on Case Studies and Experience Reports} of the Journal 
of Software Engineering for Robotics (JOSER) is dedicated to reporting
experiences from building real-world robot software systems. The primary
focus is to report experiences about the effectiveness of software 
engineering approaches and methods. A secondary focus is on reporting
general issues and problems with software engineering in robotics, 
particularly but not only for novel application domains (of which robotics
has many).

Papers for this section are fairly short (4 pages). Therefore,
while the relevant topics for this section are generally the same as for JOSER
overall, we would expect that the methods applied are not novel,
but that the focus is on case studies applying existing concepts and tools, and
reporting their experiences with them. In particular, we encourage reports on
\begin{itemize}
\item Reports \emph{confirming} the (degree of) suitability of particular method and/or concept
\item Reports \emph{identifying} a specific gap in existing methods/concepts
\item Reports \emph{stimulating} general investigation of novel methods because of unsolved issues
\end{itemize}

General information on preparing papers for JOSER can be found in~\cite{JOSER:Brugali}. The 
online version of this document, with \LaTeX{} sources, can be found at \url{https://github.com/iluetkeb/joser_case_study}. The original \LaTeX{} document for IEEE CS journals by Michael
Shell might also be useful, see \url{http://www.michaelshell.org/tex/ieeetran/}.

The remainder of this document, starting with the Introduction section, \textbf{defines a common
structure for case study papers}. It is based upon general guidelines for reporting of case 
studies, by Runeson and Höst~\cite{ESE2009:Runeson}. Please stick to this structure
as much as possible, to make case study papers more accessible.

\section{Introduction}

% you are encouraged to leave these headings in your document, to start the corresponding
% paragraph. they provide a fairly space-efficient heading.
\textbf{Problem statement}

\noindent\textbf{Research objectives}

\noindent\textbf{Context}

\section{Related Work}

\textbf{Earlier studies}

\noindent\textbf{Theory}

\section{Study Design}

\noindent\textbf{Research questions}

\noindent\textbf{Case and subjects selection}

\noindent\textbf{Data collection procedures}

\noindent\textbf{Analysis procedures}

\noindent\textbf{Validity procedures}

\section{Results}

\noindent\textbf{Case and subjects description}

This should cover execution, analysis and interpretation issues. 

Subsections here are determined by the chosen study method, linking observations to conclusions.

\noindent\textbf{Evaluation of validity}

\section{Conclusions and Future Work}

% remove this heading when adding your text
\noindent\textbf{Summary of conclusions}

\noindent\textbf{Relation to existing evidence}

\noindent\textbf{Impact/implications}

\noindent\textbf{Limitations}

\noindent\textbf{Future Work}

\section*{Acknowledgments}

% references section
\bibliographystyle{IEEEtran}
% argument is your BibTeX string definitions and bibliography database(s)
\bibliography{joser_bibliography}


% biography section
\begin{IEEEbiography}[{author}]{First Author}
bio1
\end{IEEEbiography}

\begin{IEEEbiography}[{author}]{Second Author}
bio2
\end{IEEEbiography}


\begin{IEEEbiography}[{author}]{Third Author}
bio3
\end{IEEEbiography}

% insert where needed to balance the two columns on the last page with
% biographies

% You can push biographies down or up by placing
% a \vfill before or after them. The appropriate
% use of \vfill depends on what kind of text is
% on the last page and whether or not the columns
% are being equalized.

\vfill

% Can be used to pull up biographies so that the bottom of the last one
% is flush with the other column.
%\enlargethispage{-5in}

% that's all folks
\end{document}
